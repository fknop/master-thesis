% !TEX root = ../thesis.tex

\documentclass[../thesis.tex]{subfiles}



\begin{document}

This chapter presents the resource allocation problem.
We first introduce the general problem and its constraints, 
the formal models are then described in \autoref{chapter:models}.

The resource allocation problem described in this thesis is based on the needs of the \vone\ company.
The problem consists of allocating resources to work demands.
This problem is a type of staff scheduling problem, it can be seen as a variant of the 
well known \emph{Nurses Scheduling Problem} (NSP) \cite{Burke2004}. The goal of the NSP is to assign 
nurses to shifts such that the entire schedule is satisfied. 
This type of problem often has hard constraints to state restrictions 
and soft constraints to state preferences. 


In our problem, we have a number of time periods all equal in time and work demands with multi-skill requirements
happening in those periods. 
A demand can occur in multiple time periods and consists of a required number of workers, an eventual work location 
and additional resources like machines.

More formally, the problem consists of:

\begin{itemize}
  \item[$-$] A list of multiple time periods all equal in time ($t \in T$).
  \item[$-$] A list of clients ($c \in C$). 
  \item[$-$] A list of demands ($d \in D$).
  \item[$-$] A list of workers ($w \in W$). 
  \item[$-$] A list of skills ($s \in S$).
  \item[$-$] A list of locations ($l \in L$).
  \item[$-$] A list of machines ($m \in M$).
  \item[$-$] A list of working requirements associated with workers ($r \in R$). A requirement $r$ states that a worker $r_w$
  needs to work a minimum of $r_{min}$ and maximum of $r_{max}$ times in the problem time window.  
  \item[$-$] A list of incompatibilities between workers ($(w_1, w_2) \in I_{ww}$).
  \item[$-$] A list of incompatibilities between workers and clients ($(w, c) \in I_{wc}$).
\end{itemize}

Each demand has:

\begin{itemize}
  \item[$-$] A client ($d^c \in C$).
  \item[$-$] A given set of time periods ($d^T \subseteq T$).
  \item[$-$] A required number of workers per period ($d^w \in \mathbb{N}$).
  \item[$-$] Some skills requirements to be fulfilled by different workers ($d^S \subseteq S$).
  It imposes that some workers have the needed capacities to work at a given position (e.g. package lifter).
  \item[$-$] Additional skills requirements to be fulfilled by any workers assigned to that demand (e.g. driver license) ($d^{S^{+}} \subseteq S$).
  \item[$-$] A list of machines to perform the work ($d^M \subseteq M$).
  \item[$-$] An eventual list of possible locations where the demand can be performed ($d^L \subseteq L$). 
  Vehicles used to drive the workers to the work location are considered as machines.
\end{itemize}


Each worker has:

\begin{itemize}
  \item[$-$] A list of skills (e.g. package lifter, supervisor, etc.) ($w^S \subseteq S$).
  \item[$-$] A list of availabilities at which the worker can work ($w^T \subseteq T$).  
\end{itemize}


The goal is to assign workers to multi-skill positions as well as machines and locations to a list of demands over the set of all time slots.
Each resource can only be assigned once per time period and needs to satisfy all the constraints stated by the demand. 
% The possible objective is to assign workers in such a way that they work for the longest time possible at the same position.

\section{Constraints}
\subsection{Hard Constraints}

\subsubsection{A worker can only work when available}

Each worker has a defined set of availabilities and cannot be assigned to a demand when 
unavailable.

\subsubsection{No worker should be assigned to a demand which is not occurring}

A demand has a set of time periods in which it occurs,
no workers should be assigned to that demand if the demand is not 
occurring.

\subsubsection{No worker can be assigned twice for the same period} 
A worker cannot do the work of two different workers at the same time.
Hence, a worker can only work at most once per time period.

\subsubsection{Each demand has a required number of workers}

Each demand needs a number of workers to be satisfied. 
For each time period in which a demand is occurring, it should have the required 
number of workers assigned to it.

\subsubsection{Each assignment must respect skill restrictions}

Each position of a demand might require skills to be satisfied. 
To be assigned to that position, a worker must have the required skills. 
A worker can also have more skills than the required skills by the position.

\subsubsection{Worker-worker incompatibilities}

Workers might be incompatible with each other. Such workers cannot
be assigned together at the same time period.

\subsubsection{Worker-client incompatibilities}

A worker and a client might be incompatible with each other. 
If this is the case, the worker must not be assigned at a demand for such client.

\subsubsection{The required machines must always be assigned}

A demand has machine needs. Such machines should always be assigned 
for a demand to be satisfied.

\subsubsection{No machines should be assigned twice for the same period}

A machine is assigned for the entirety of a demand. It can be used for other demands 
that do not overlap in time with the first one. But it can never be assigned twice 
for the same time period.

\subsubsection{The location assigned must be in the set of possible locations}

A demand has a set of possible locations. Only one of those locations can be assigned 
to that demand.

\subsubsection{No location should be assigned twice for the same period}

As with machines, locations must be assigned only once per time period.


\subsection{Soft Constraints}

\subsubsection{Satisfy the most assignments possible}

Each demand need a required number of workers. However, we can assign a fictitious worker to demands 
and minimize the number of fictitious workers. This is done in the case where there is not enough workers
or no worker that satisfy a particular skill. From a modeling point of view, the hard constraint which states
that the required number of workers must be satisfied is still satisfied with a fictitious worker.

\subsubsection{Contiguous shifts}

A demand consists of multiple positions over a period of time. 
For each position, a worker should keep working at that position for the longest time possible. 
We want to avoid the hassle of changing shift every time. 
As this constraint is harder to solve, we express it as a soft constraint and minimize the number of 
violations.

\subsubsection{Working requirements}

Workers can have minimum and maximum working periods. We want to make sure 
that these requirements are satisfied. However, as this is not always possible to solve, we 
state this as a soft constraint.

\end{document}

