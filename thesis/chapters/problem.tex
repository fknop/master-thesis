% !TEX root = ../thesis.tex

\documentclass[../thesis.tex]{subfiles}



\begin{document}

This chapter presents the resource allocation problem.
We first introduce the general problem and its constraints, 
the formal models are described in \autoref{chapter:models}.

The \vone\ problem consists of allocating resources to work demands.
This problem is a type of staff scheduling problem, it can be seen as a variant of the 
well known \emph{Nurses Scheduling Problem} (NSP) \cite{Burke2004}. The goal of the NSP is to assign 
nurses to shifts such that the entire schedule is satisfied. 
This type of problem often have hard constraints to state restrictions 
and soft constraints to state preferences. 

\vone\ has internal work for their employees but also receives 
external labor requests. The problem is separated in multiple time periods all equal in time.
A demand often occurs in multiple time slots and consists of a required number of workers, an eventual 
work location and additional resources like machines or vehicles. Each demand has:

\begin{itemize}
  \item A given set of time periods.
  \item A required number of workers per period.
  \item Some skills requirements to be fullfilled by the workers. 
  It imposes that some workers have the needed capacities to work at a given position (e.g. package lifter).
  \item A list of machines to perform the work.
  \item An eventual list of possible locations where the demand can be executed and a vehicle to drive the workers to destination.
  A demand can only have a location if it is an external labor request. Internal work to the company use predefined locations.
  \item An eventual need for a worker supervisor which will supervise the group.
\end{itemize}


Each worker has:

\begin{itemize}
  \item Some skills and restrictions (e.g. package lifter, supervisor, etc.)
  \item A list of availabilities at which the worker can work.
  \item A list of incompatibilities with other workers (i.e. workers that can't work together).
  \item A list of incompatibilities with clients (i.e. workers that can't work for clients).
\end{itemize}

The goal is to assign workers, machines and locations to a list of demands over the set of all time slots.
Each resource can only be assigned once per time period and need to satisfy all the constraints stated by the demand. 
The sub-goal is to also assign workers in such a way that they work for the longest time possible at the same position and 
 such that the assignments between workers are balanced throughout the entire schedule.

\section{Constraints}
\subsection{Hard Constraints}

\subsubsection{Respect worker availabilities}

A worker has a set of availabilities and should not be assigned to a 
shift when not available.

\subsubsection{Respect demand occurences}

A demand has a set of time periods in which it occurs,
no workers should be assigned to that demand if the demand is not 
occuring.

\subsubsection{No worker should be assigned twice for the same period}

A worker obviously can't work at two positions at the same time. 

\subsubsection{Required number of workers}

A demand has a needed number of workers to be satisfied. 
For each time period a demand is occuring, it should have the required 
number of workers assigned to it.

\subsubsection{Skill restrictions}

Each position of a demand might require skills to be satisfied. 
To be assigned to that position, a worker must have the required skills.

\subsubsection{Worker-worker incompatibilities}

Workers might be incompatible with each other. Such workers can't 
be assigned together at the same time period.

\subsubsection{Worker-client incompatibilities}

A worker and a client might be incompatible with each other. 
If this is the case, the worker must not be assigned at a demand for such client.

\subsubsection{The required machines must always be assigned}

A demand has machine needs. Such machines should always be assigned 
for a demand to be satisfied.

\subsubsection{No machines should be assigned twice for the same period}

A machine is assigned for the entirety of a demand. It can be used for other demands 
that do not overlap in time with the first one. But it can never be assigned twice 
for the same time period.

\subsubsection{The location assigned must be in the set of possible locations}

A demand has a set of possible locations. Only one of those locations can be assigned 
to that demand.

\subsubsection{No location should be assigned twice for the same period}

As with machines, locations must be assigned only once per time period.




\subsection{Soft Constraints}

\subsubsection{Client-worker preference}

A client might prefer some workers over others. 
We use a soft constraint for this as it might not always be possible to 
satisfy.

\subsubsection{Contiguous shifts}

A demand consists of multiple positions over a period of time. 
For each position, a worker should keep working at that position for the longest time possible. 
We want to avoid the hassle of changing shift everytime. 
As this constraint is harder to solve, we express it as a soft constraint and minimize the number of 
violations.

\subsubsection{Workload balancing}

Worker assignments should be balanced throughout the entire schedule.
We want to make sure every employees has enough work.
No worker should work everyday while another barely works.

\end{document}

