% !TEX root = ../thesis.tex

\documentclass[../thesis.tex]{subfiles}



\begin{document}

This chapter presents the resource allocation problem.
We first introduce the general problem and its constraints, 
the formal models are described in \autoref{chapter:models}.

As stated before, the problem consists of allocating resources to work demands. 
The company (\vone) has internal work for their employees but also receives 
external labor requests. The problem is separated in multiple time periods all equal in time.
A demand often occurs in multiple time slots and consists of a required number of workers, an eventual 
work location and additional resources like machines or vehicles.

Each demand has:

\begin{itemize}
  \item A given set of time periods.
  \item A required number of workers per period.
  \item Some skills requirements to be fullfilled by the workers. 
  It imposes that some workers have the needed capacities to work at a given position (e.g. package lifter).
  \item A list of machines to perform the work.
  \item An eventual list of possible locations where the demand can be executed and a vehicle to drive the workers to destination.
  A demand can only have a location if it is an external labor request. Internal work to the company use predefined locations.
  \item An eventual need for a worker supervisor which will supervise the group.
\end{itemize}


Each worker has:

\begin{itemize}
  \item Some skills and restrictions (e.g. package lifter, supervisor, etc.)
  \item A list of availabilities at which the worker can work.
  \item A list of incompatibilities with other workers (i.e. workers that can't work together).
  \item A list of incompatibilities with clients (i.e. workers that can't work for clients).
\end{itemize}

The goal is to assign workers, machines and locations to a list of demands over the set of all time slots.
Each resource can only be assigned once per time period and need to satisfy all the constraints stated by the demand. 
The sub-goal is to also assign workers in such a way that they work for the longest time possible at the same position and 
 such that the assignments between workers are balanced throughout the entire schedule.


\end{document}

