% !TEX root = ../../thesis.tex

\documentclass[../../thesis.tex]{subfiles}
 
\begin{document}

\section{Nurse scheduling problem}

The nurse scheduling problem (NSP) is a well-known combinatorial problem.
It involves assigning nursing staff to shifts. It takes into account 
both hard and soft constraints. The objective maximize the preferences 
of the nursing staff while minimizing the violations of the soft constraints.
The problem is known to be NP-hard \cite{Osogami2000}.
The NSP can be transformed to many types of staff scheduling problems. 
Due to this, the literature contains a lot of different methods to solve it.

Our problem described in \autoref{chapter:problem} is a staff scheduling problem with resource allocation. 
It can be seen as a variant of the nurse scheduling problem. For example, assigning nurses to shifts can be 
transformed into assigning workers to demand positions and assigning patients to rooms can be 
seen as assigning locations to demands.


\end{document}

