% !TEX root = ../../thesis.tex

\documentclass[../../thesis.tex]{subfiles}
 
\begin{document}

A \emph{Constraint Satisfaction Problem} (CSP) consists of a set of $n$ variable, 
$\{x_1, \dots, x_n \}$; a domain $D(x_i)$ of possible values for each variable $x_i$, 
$1 \leq i \leq n$; and a collection of $m$ constraints $\{ C_1, \dots, C_m \}$. 
Each constraint $C_j$, $1 \leq j \leq m$, is a constraint over some set of variables called the scheme 
of the constraint. The size of this set is known as the arity of the constraint. 
A solution to a CSP is an assignment of values $a_i \in D_i$ to $x_i$, that satisfies all of the constraints. \cite{cp-definition}

\subsection{Global Constraints}

As described in more depth in \cite{Hentenryck:2003}:

\begin{quotation}
  [\dots] a constraint $C$ is often called “global” when “processing” $C$ as a whole gives better results than “processing” any conjunction
  of constraints that is “semantically equivalent” to $C$.
\end{quotation}

The author also define three types of constraint globality, we are mostly interested in what he refers to \emph{operatinal globality}. 
Those constraints can be decomposed into multiple simpler constraints but the filtering quality of the decomposition
is often worse than its global counterpart. 

There also exists soft variants of global constraints where those constraints are associated with 
a number of violations which is usually minimized afterwards. This is particularely useful for problems with impossible 
solution while using the hard constraints version.

\subsubsection{\texttt{alldifferent} constraint}

The \texttt{allddifferent} constraint \cite{Rgin1994AFA} is one of of the most famous global constraint used in Constraint 
Programming.
This constraint is defined over a subset of variables for which values must be different. More formally:

\begin{equation*}
  \texttt{alldifferent}(x_1, \dots, x_n) = \{ (d_1, \dots, d_n) \mid d_i \in D(x_i), d_i \neq d_j \forall i \neq j \}
\end{equation*}

This constraint can be decomposed into multiple binary inequalities. It makes \texttt{alldifferent} an operational global constraint.
It can be proven that the filtering of the global constraint cannot be achieved with a decomposition. As 
an example, let's define three variables $x_1$, $x_2$ and $x_3$ respectively taking domains $\{1,2\}$, $\{1,2\}$, $\{1,2,3,4\}$. 
The global constraint would be able to successfuly filter $1$ and $2$ from the domain of 
$x_3$ because the values are always taken by $x_1$ and $x_2$. However, the decomposition is not able to filter those values.


\subsubsection{Global Cardinality Constraint}

The global cardility constraint (\texttt{gcc}) \cite{Regin:1996} is a generalization of the 
\texttt{alldifferent} constraint. It does not enforces (although it can) the uniqueness of values of its variables
but instead enforces that the cardinality of each value $d_i$ for all its variables in its scope lies
between a lowerbound and a upperbound, respectively $l_i$ and $u_i$. 

\begin{equation*}
  \texttt{gcc}(X, l, u) = \{ (d_1, \dots, d_n) \mid d_i \in D(x_i), l_d \leq |\{ d_i \mid d_i = d \}| \leq u_d, \forall d \in D(X) \}
\end{equation*}

As stated above, we can express the \texttt{alldifferent} constraint with this definition:

\begin{equation*}
\texttt{gcc}(\{ x_1, \dots, x_n \}, [1, \dots, 1], [1, \dots, 1])
\end{equation*}


We are also interested in a soft variant of \texttt{gcc} called \texttt{softgcc} \cite{VanHoeve2006}. 
The violation associated with this constraint is the sum of excess or shortage \cite{schaus:softgcc} for each value.

\begin{align*}
  \texttt{softgcc}(X, l, u, Z) &= \{ (d_1, \dots, d_n) \mid d_i \in D(x_i), d_z \in D(Z), viol(d_1, \dots, d_n) \leq d_z \} \\
  \text{with} \quad viol(d_1, \dots, d_n) &= \sum_{d \in D(X)} \text{max}(0, |\{ d_i \mid d_i = d \}| - u_d, l_d - |\{ d_i \mid d_i = d \}|)
\end{align*}



\end{document}

