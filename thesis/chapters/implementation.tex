% !TEX root = ../thesis.tex

\documentclass[../thesis.tex]{subfiles}
 
\begin{document}

In this chapter, we describe our implementation for the models presented in \autoref{chapter:models}. 
We talk about the difficulties encountered while trying to transform the theoretical model to code. 
We will also discuss some differences between the models and the implementation and some trade-offs that were taken in order to 
have the most performant solver.

The implementation was done in Scala using 
\emph{OscaR} (\ref{subsection:oscar}) as CP solver and
\emph{Gurobi Optimizer} (\ref{subsection:gurobi}) as MIP solver. 



\section{Input and output format}

For consistency, both solvers take the same input format and returns the same output format.
We created a JSON (JavaScript Object Notation) Schema \cite{json:schema} to formulate our problems and solution assignments.
Those schemas allow us to create a typed data structure for JSON objects. All the typing validation 
is handled by the JSON Schema library. A small example of JSON schema can be found in \ref{json:example}.

\begin{lstlisting}[language=json,firstnumber=1,caption={JSON Schema example},captionpos=b,label={json:example}]
  "client": {
    "type": "object",
    "properties": {
      "name": {
        "type": "string"
      }
    },
    "required": ["name"]
  }
\end{lstlisting}

\section{Mixed Integer Programming solver}

We used the Java API \cite{gurobi:java} of the Gurobi Optimizer in Scala to create our implementation. 
The solver was implemented almost exactly like the theoretical model.

\section{Constraint Programming solver}





\end{document}

