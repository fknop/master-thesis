% !TEX root = ../thesis.tex

\documentclass[../thesis.tex]{subfiles}
 
\begin{document}


\section{Constraint Programming Model}

The translation to the mathematical (MIP) model to the CP model is fairly
straightforward. Binary variables are translated to integer variables, each value representing
one resource (i.e. worker, zone or machines).



\subsection{Variables}

First, we need to express the set of workers for each demand at each time period in which that demand occurs.
This is done by using a 3-dimensional array of variables. The first dimension being the indices of the 
time periods, the second dimension is the indices of the demands while the last dimension is the list of worker variables. 
This last dimension has the size of the number of required workers for that demand.

\begin{equation}
\begin{split}
    w[i][j][k] \in W \label{cpworkervariable} \\
\end{split}
\end{equation}

(\ref{cpworkervariable}) is the worker working at time $i$ for demand $j$ at position $k$ with $t_i \in T$, $d_i \in D$ and $t_i \in d_j^T$.

The same reasoning is used for zones and machines:

\begin{align}
    m[i][j] &\in M \label{cpmachinevariable} \\ 
    z[i] &\in Z \label{cpzonevariable} 
\end{align}

(\ref{cpmachinevariable}) is the $j^{\text{th}}$ machine used for demand $i$ while (\ref{cpzonevariable}) is the zone used for demand $i$


\end{document}

