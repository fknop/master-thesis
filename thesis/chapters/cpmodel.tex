% !TEX root = ../thesis.tex

\documentclass[../thesis.tex]{subfiles}
 
\begin{document}


\section{Constraint Programming Model}

The translation to the mathematical (MIP) model to the CP model is fairly
straightforward. Binary variables are translated to integer variables, each value representing
one resource (i.e. worker, zone or machines).



\subsection{Variables}

First, we need to express the set of workers for each demand at each time period in which that demand occurs.

\begin{equation}
\begin{split}
    w_{ijk} \in W \label{cpworkervariable} \\
\end{split}
\end{equation}

(\ref{cpworkervariable}) is the worker working at time $i$ for demand $j$ at the $k^{\text{th}}$ position 
with $t_i \in T$, $d_i \in D$, $t_i \in d_j^T$ and $k \in d^P_j$. This is done by using a 3-dimensional array of variables. The first dimension being the indices of the 
time periods, the second dimension is the indices of the demands while the last dimension is the list of worker variables. 
This last dimension has the size of the number of required workers for that demand.

The same reasoning is used for zones and machines:

\begin{align}
    m_{ij} &\in M \label{cpmachinevariable} \\ 
    z_i &\in Z \label{cpzonevariable} 
\end{align}

(\ref{cpmachinevariable}) is the $j^{\text{th}}$ machine used for demand $i$ while (\ref{cpzonevariable}) is the zone used for demand $i$

Some constraints are already satisfied by the modeling of the variables, like the number of required resources (i.e. worker, location, machine)
per demand.

\subsection{Constraints}

\subsubsection{A worker can only work for one demand at a time}

Let $X_i = \{w_{ijk} \mid j \in D, k \in d_j^P \}$ be the set of worker variables 
for all the demands accross time period $i$. The constraint $\texttt{allDifferent}(X_i)$ states that all workers working at time $i$ need to be different.
For each time period $i$ and set $X_i$, we need an \texttt{allDifferent} constraint.

\subsubsection{All workers can only work when they’re available}

Let $W_i = \{w \mid t_i \notin w^T \}$ be the set of workers not available at $t_i$. 
The constraint \texttt{notEqual} can be used to remove values from the worker variables $w_{ijk}$.
For each time period $i$ and demand, remove the values contained in the set $W_i$.
This can however be done at the domain initialization time by omitting those values instead of having an
additional constraint.

\subsubsection{Incompatibilities between workers}

We can solve this constraint using a \texttt{negativeTable} constraint. 
For each pair of variables $(a, b)$ in the set $\{w_{ij0}, \dots, w_{ijn \mid n = d^w_j - 1}\}$, we add $\texttt{negativeTable}(a, b, I_{ww})$.
This prevents workers to work with incompatible other workers. Note that we need to cover the two directions of incompatibilities from $I_ww$. 
We can do this by adding another $\texttt{negativeTable}(b, a, I_{ww})$ or simply by appending the reversed direction to the $I_{ww}$ table.

\subsubsection{Incompatibilities between workers and clients}

This constraint is much simpler than the one described above. Contrary to the worker-worker incompatibilities,
the clients are fixed values. 
We simply need to remove values 
from the domain of worker variables. For each $(w, c) \in I_{wc}$, we remove the value $w$ to each variable on 
demand $d$ where $d^c = c$. This can be done by a \texttt{notEqual} constraint or by removing the value at initialization.

\subsubsection{Skill requirements}

The variable modeling makes this constraint simple. Each position is described by one worker variable.
Each position is assigned a set of skills needed by one worker. Again, we need to remove values 
from the domain of the worker variables when those workers do not meet the skill requirements of that position. 

For example, if a demand has a requirement of two workers, one \texttt{lifter} and another one with no particular skill. 
The first worker variable for that demand will remove all workers that do not have the \texttt{lifter} skill while the second variable will remain untouched.

\subsubsection{Additional skill requirement}

Additionnal skills are the skills that can be satisfied by any worker in the group.
TODO: explain gcc + sum occurences

\subsubsection{No zone should be used by two overlapping demands}

Let $Z^O_i = \{ z_j \mid j \in d^O_i \}$ be the set of zone variables for demands that overlap in time with demand $i$.
The constraint $\texttt{allDifferent}(Z^O_i)$ states that all zones for overlapping demands 
should be different. For each demand $i$ and set $Z^O_i$, we add an \texttt{allDifferent} constraint.

\subsubsection{No machine should be used by two overlapping demands}

Let $M^O_i = \left\{ m_{jk} \mid j \in d^O_i, k \in \{0, \dots, |d^M_j| - 1 \} \right\}$ be the set of machine variables for demands that overlap in time 
with demand $i$. The constraint $\texttt{allDifferent}(M^O_i)$ states that all machines for overlapping demands 
should be different. For each demand $i$ and set $M^O_i$, we add an \texttt{allDifferent} constraint.

\subsubsection{Objective function}

The objective function expresses the minimization of the sum of different worker for each shift.
Let $W_{jk} = \{ w_{ijk} \mid i \in d^T_j \}$ be the set of all worker variables for demand $j$ at position $k$ accross all time periods for that demand.
We use this set to compute the number of different workers for a given shift with the constraint
$\texttt{atLeastNValue}(W_{jk}, N_{jk})$ with $N_{jk}$ being the number of different workers for shift $k$ of demand $j$.
The objective can be expressed by $\sum_{j, k} N_{jk}$ which is the sum of different workers over all shifts. 

\end{document}

