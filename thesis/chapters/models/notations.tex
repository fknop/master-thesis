% !TEX root = ../../thesis.tex

\documentclass[../../thesis.tex]{subfiles}
 
\begin{document}

% TODO: REFORMAT THIS
\section{Notations}

\begin{itemize}

  \item[--]
  Set of periods:
  \begin{equation*}
    T = \{ 0, \dots, n \mid n \in \mathbb{N} \}
  \end{equation*}

  \item[--]
Set of workers:
  \begin{equation*}
      W = \{ w_0, \dots, w_n \mid n \in \mathbb{N} \}
  \end{equation*} 
  \begin{itemize}
    \item $w^T \subseteq T$: Availabilities of a worker:
  \end{itemize}

  \item[--] Set of machines:
  \begin{equation*}
      M = \{m_0, \dots, m_n \mid n \in \mathbb{N}, m_n \in \mathbb{N} \}
    \end{equation*}
    
    Let's also define the set of machines for a given machine value (i.e name). $M_i$ is the set of machines that takes the value $i$. 
    
    \begin{equation*}
        M_i = \{m_j \mid m_j = i,  \forall j \in M\}
    \end{equation*}

  \item[--] Set of vehicles:
  \begin{equation*}
      V = \{v_0, \dots, v_n \mid n \in \mathbb{N} \}
  \end{equation*}
  This could also be expressed as a subset of machines:
  
  \begin{equation*}
      V \subseteq M
  \end{equation*}
\
  \item[--] Set of zones:
  \begin{equation*}
      Z = \{z_0, \dots, z_n \mid n \in \mathbb{N} \}
  \end{equation*}

  \item[--] Set of demands: 
  \begin{equation*}
      D = \{ d_0, \dots, d_n \mid n \in \mathbb{N} \}
  \end{equation*}

  \begin{itemize}
    \item $d^w \in \mathbb{N}$: Required number of workers for this demand
    \item $d^T \subseteq T$: Possible periods for a demand
    \item $d^Z \subseteq Z$: Possible zones for a demand
    \item $d^M \subseteq M$: List of required machines by the demand
    \item $d^c \in C$: Client for that demand
    \item $d^S \in S$: List of skill required by the demand (each skill need to have a different worker)
    \item $d^{s_0}$: The first skill in $d^S$
    \item $d^{S^+}$: Set of additional skills that can be satisfied by any worker in that demand
    \item $d^{S^+_0}$: The first skill in $d^{S^+}$
    \item $d^P \in \{ 0, \dots, d^w - 1 \}$: List of positions
    \item $d^O \subseteq D$: Set of overlapping demands in time for that demand. e.g. the overlapping demands for demand 1 is $d_1^O$
  \end{itemize} 

  Let's also define the set of demands where the client $c$: ${D_c = \{ d \mid d^c = c \}}$

  \item[--] Set of clients: 
  \begin{equation*}
      C = \{ c_0, \dots, c_n \mid n \in \mathbb{N} \}
  \end{equation*}

  \item[--] Set of skills: 
  \begin{equation*}
      S = \{ s_0, \dots, s_n \mid n \in \mathbb{N} \}
  \end{equation*}
  Let's also define the set of workers that satisfy a skill or skill set:
  
  \begin{equation*}
      W_s \subseteq W, s \in S
  \end{equation*}

  \item[--] Set of working requirements: 
  \begin{equation*}
      R = \{ r_0, \dots, r_n \mid n \in \mathbb{N} \}
  \end{equation*}

  \begin{itemize}
    \item $r_{w}$: The worker concerned with this requirement
    \item$r_{min}$: The minimum number of periods the worker has to work 
    \item $r_{max}$: The maximum number of periods the worker has to work
  \end{itemize} 
  
  \item[--] Set of incompatibilities between workers:
  \begin{equation*}
      {I_{ww} = \{ ({i},{j}) \in \mathbb{N} \times \mathbb{N} \mid w_i, w_j \in W, w_i \neq w_j \}}
  \end{equation*}

  \item[--] Set of incompatibilities between workers and clients:
  \begin{equation*}
      I_{wc} = \{ (i, j) \in \mathbb{N} \times \mathbb{N} \mid w_i \in W, c_j \in C \}
  \end{equation*}
\end{itemize}  


\end{document}

