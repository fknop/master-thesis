% !TEX root = ../thesis.tex

\documentclass[../thesis.tex]{subfiles}
 
\begin{document}


Staff scheduling is an important problem encountered by many organizations, such as industry or hospitals.
One of the most studied staff scheduling problem is the nurse scheduling problem (NSP). 
Staff scheduling problems such as the NSP become hard to solve as the problem grows in size due to their NP-hard complexity.

This thesis presents an optimization problem for production planning with resource allocation based on the needs of a Belgian company called \vone.

Village n\textsuperscript{o}1 employs people with disabilities.
They offer services to companies and private individuals such as industrial jobs. They
are currently in the process of automating the way they schedule these jobs. 

Due to the important nature of staff scheduling, a lot of techniques exist to solve these problems. 
The aim of this thesis is to present two approaches that solve the problem described 
in \autoref{chapter:problem}. These two approaches are Constraint Programming and Mixed Integer Programming.
We then analyze and discuss the performances of these approaches.



This thesis is organized as follows:

\begin{itemize}
  \item[] \autoref{chapter:problem} introduces the resource allocation problem derived from the needs of \vone.
  \item[] \autoref{chapter:sota} describes the state-of-the-art in the domains of Mixed Integer Programming and Constraint Programming. 
  \item[] \autoref{chapter:models} gives formal definitions of the Mixed Integer Programming and the Constraint Programming models.
  \item[] \autoref{chapter:implementation} describes the implementation of the models presented in \autoref{chapter:models}.
  \item[] \autoref{chapter:experiments} presents the carried experiments, their results, and our analysis of these results.
  \item[] \autoref{chapter:conclusion} concludes this thesis by giving some takeaway and suggests some improvements that could be made to the solvers.
\end{itemize}

\end{document}

